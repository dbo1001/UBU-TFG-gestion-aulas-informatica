\capitulo{3}{Conceptos teóricos}
En este apartado se explican diferentes conceptos teóricos relacionados con el proyecto.

\section{\textit{Framework}}
Un \textbf{\textit{framework}}~\cite{framework_definicion}, o marco de trabajo, es una herramienta software que proporciona una base de código y formas, estandarizadas y consistentes para el desarrollo de aplicaciones web.

\section{\textit{Frontend}}
El \textbf{\textit{frontend}}~\cite{frontend_backend} es con lo que el usuario interactúa en una web, es la parte a la que se tiene acceso directamente. 

Los lenguajes de desarrollo de \textit{frontend} más comunes son HTML, CSS y JavaScript, aunque también se puede hacer uso de frameworks y librerías como React, Angular, Bootstrap \dots

\section{\textit{Backend}}
El \textbf{\textit{backend}}~\cite{frontend_backend} es la capa de acceso a los datos, que se conecta con la base de datos y el servidor utilizados en la web, y que no es accesible por los usuarios, ya que contiene la parte lógica de la aplicación.

Los lenguajes de desarrollo de \textit{backend} más comunes son Java, Python, Ruby \dots

\section{Desarrollo \textit{full stack}}
El desarrollo \textbf{\textit{full stack}}~\cite{desarrollo_full_stack} es aquel que engloba tanto la producción del lado del cliente o \textit{frontend} como la del lado del servidor o \textit{backend}.

\section{Convención sobre configuración} \label{convencion_sobre_configuracion}
La \textbf{\textit{Convención sobre Configuración}}~\cite{wiki:convencion_sobre_configuracion} (CoC) es un paradigma de programación software con el cual se tratan de minimizar las decisiones que debe tomar el desarrollador, simplificando sin perder flexibilidad.