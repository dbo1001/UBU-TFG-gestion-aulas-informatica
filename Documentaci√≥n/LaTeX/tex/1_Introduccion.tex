\capitulo{1}{Introducción}

Con fin de solventar los problemas que tienen los profesores a la hora de reservar aulas para la realización de exámenes, o ver si se encuentran disponibles, sin necesidad de desplazarse o consultar papeles que puede que no estén al día, ha surgido este proyecto centrado en desarrollar aplicación web para llevar a cabo su gestión.

Lo que se pretende con el mismo es permitir a cualquier usuario consultar tanto las reservas registradas como la disponibilidad de las aulas. De esta forma, un profesor que esté planificando su examen puede ver qué horario y aula le convienen, consultando si está disponible en una fecha y hora específicas o filtrando a través de ciertos parámetros (capacidad, número de ordenadores, centro en el que se encuentra, etc.).

Una vez haya decidido la fecha, hora y lugar de la reserva, acudirá al responsable del departamento que se encuentra a cargo del aula, o en su defecto al responsable del centro, para pedirle que la registre. Si por algún casual esta reserva se tiene que modificar, o incluso eliminar, deberá acudir de nuevo a él.