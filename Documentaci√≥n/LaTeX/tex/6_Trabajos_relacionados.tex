\capitulo{6}{Trabajos relacionados}
En este apartado se comentarán aplicaciones y trabajos de carácter similar, relacionados con las reservas.

Una de las ventajas del presente proyecto es que está enfocado a la gestión de reservas de los espacios de la universidad, ya que se trabajan con detalles como la capacidad de las aulas o el número de ordenadores que hay en las mismas.

Otro detalle que cabe destacar es que se permite a cualquier usuario consultar las reservas y la disponibilidad de las aulas, pudiéndose realizar un filtrado por fechas, horas, días, centros, aulas, etc. para obtener búsquedas más concretas y detalladas.

Además, otra de las ventajas es que la reserva de aulas queda restringida al responsable del centro o departamento propietario del aula que se quiere reservar, de forma que no se satura el sistema con peticiones de reserva de cualquier persona para su aprobación.

\section{Sistema Interno de Reserva de Espacios de la EPS de la Universidad de Sevilla}
El propio Laboratiorio de Informática de la Escuela Politécnica Superior de la Universidad de Sevilla ha desarrollado una aplicación para gestionar los espacios de la propia escuela~\cite{pagina_eps_sevilla}.

\section{Booked}
Booked~\cite{pagina_booked} es un programador de reservas de pago disponible en 40 idiomas, que permite gestionar las reservas de aulas, salas de reuniones, etc., los recursos (aulas y salas que se pueden reservas), y visualizar las reservas en un calendario.

\section{SuperSaaS}
SuperSaas~\cite{pagina_supersaas} es un planificador de reservas online. Como indican en su web está adaptada a acualquier tipo de negocio permitiendo programación uno a uno (para terapeutas, entrenadores personales, etc), programación de grupo (gimnasios, turismo, escuelas, etc.), reservas y alquileres (alquiler de espacios, planificación de recursos) y reserva de citas (enfocada a salones de belleza, profesionales, peluquerías, etc). Dispone además de una opción gratuita para fines no comerciales.

Una parte interesante de la misma es que permite su integración en el sitio web propio y en Facebook a través de su API y tiene integrado un sistema de pagos.

\section{Acuity Scheduling}
Acuity Scheduling~\cite{pagina_acuity} es un asistente personal de la agenda enfocado a empresas de fitness que ofertan clases y sesiones privadas, permitiendo reservar citas, consultar la disponibilidad en tiempo real y el pago por adelantado. Tiene disponible también una versión gratuita y ofrece un periodo de prueba gratis.